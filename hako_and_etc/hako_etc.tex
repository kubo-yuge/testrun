\documentclass[a4paper,10pt,oneside,notitlepage,final]{jsarticle}
\usepackage{amsmath,amssymb,multicol,fancyhdr}  % AMSパッケージ,多段組パッケージ,ヘッダ制御の追加
\usepackage{ketpic,ketlayer,graphicx,color} % ketpic & color の利用


\begin{document}
\vspace*{-10mm}
{\large \verb|\hako|コマンドの動作確認}\\ % タイトル

\Large

\newcommand{\hako}[4][6]{\fbox{\raisebox{#2 mm}{$\mathstrut$}\raisebox{-#3 mm}{$\mathstrut$}\Ctab{#1 mm}{#4}}}

高遠先生の\verb|\hako|を確認する。(中の式の高さが大きいと枠が伸びる)\\[-5mm]

\begin{layer}{80}{25}
\end{layer}

\hako[50]{10}{2}{\Huge{$\dfrac{Yg}{2}$}a1y}aaa
\rule[0pt]{25mm}{.1pt}基準線\\


\vspace{15mm}
以下では,中の式の高さに関わらず指定した高さの枠を用意するコマンドを定義。\\

%% \fixbox の定義
\newcommand{\fixbox}[5][c]
{
  \fboxsep=0pt \framebox[#2][#1]
  {
   \rule[-#4]{0mm}{#4}\rule[0mm]{0mm}{#3}
   \smash{#5}$\mathstrut$% \mathstrutを最低値とする
  }
}

\verb|\fixbox|を定義:\\
\verb|\fixbox[横配置]{幅}{箱の高さ}{下に移動}{文字列}|\\
\verb|\fixbox[c]{50mm}{10mm}{2mm}{文字列}|\\[-2mm]

\begin{layer}{80}{15}
\end{layer}

\fixbox[c]{50mm}{10mm}{2mm}{\Huge{$\dfrac{Yg}{2}$}a2y}aaa
\rule[0pt]{25mm}{.1pt}基準線\\

\vspace{10mm}
%% \fixboxm の定義
\newcommand{\fixboxm}[5][c]
{
  \makebox[#2][#1]
  {
   \rule[-#4]{0mm}{#4}\rule[0mm]{0mm}{#3}
   \smash{#5}$\mathstrut$% \mathstrutを最低値とする
  }
}

\verb|\fixboxm|を定義:(枠線を引かない)\\
\verb|\fixboxm[横配置]{幅}{箱の高さ}{下に移動}{文字列}|\\
\verb|\fixboxm[c]{50mm}{10mm}{2mm}{文字列}|\\[-2mm]

\begin{layer}{80}{15}
\end{layer}

\fixboxm[c]{50mm}{10mm}{2mm}{\Huge{$\dfrac{Yg}{2}$}a3y}aaa
\rule[0pt]{25mm}{.1pt}基準線\\


\end{document}