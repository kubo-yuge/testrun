%\documentclass[dvipdfmx,a4paper,twoside]{jsarticle}
\documentclass[uplatex,dvipdfmx,a4paper,twoside]{jsarticle}
\input{弓削紀要様式R1.tex} % 弓削紀要の様式(タイトル、番号)に合わせるためのもの
% upLaTeXでコンパイルするように変更(昨年までと異なる)。

\usepackage[bookmarks=false,colorlinks=true,urlcolor=blue]{hyperref}
\usepackage{url}

\newcommand{\maintitlej}{{\ketcindy {の紹介}}} % ここで和文タイトルを定義して使い回す.

\begin{document}

%%%% 最初のページのヘッダ・フッタ(ここから) %%%% ヘッダ無し、フッタあり。(フッタ:所属と提出日)
\thispagestyle{empty}

\begin{center}
\vspace*{-6mm}
{\fontsize{21pt}{0pt}\selectfont \textbf{\maintitlej}} % 和文表題を\maintitlejで挿入

% {\fontsize{16pt}{0pt}\selectfont --- 平成30年度の準備(副題16pt) ---} % 和文副題
 \end{center}
\vspace*{.3\baselineskip} % centring による行送りを考慮して、10ポイントの1行空きに調整する。

{\setlength{\baselineskip}{19pt}
\begin{center} {\fontsize{17pt}{0pt}\selectfont\bf 
--- 点や直線を動かしてみよう ---
} % 英文タイトル

%{\fontsize{14pt}{0pt}\selectfont --- Sub Title(14point, centering) ---} % 英文サブタイトル
\end{center} }
\vspace*{.3\baselineskip}


\vspace*{2\baselineskip}
%%%% ここまででタイトル・著者・Abstract など本文に入る前の準備をすます。%%%%

\begin{multicols}{2}
\vspace*{-2.5\baselineskip} % 2段組の開始がタイトルなので、右の段に会わせるための調整
\section{直線をはさむ2点}

\subsection{2点P,Q は対称}

直線をはさむ2点を見てみよう。\par
\begin{layer}{80}{0}
\putnotese{7}{7}{%%% /Users/kumac21/DataKubo_Mac21/予定とメモなど/予定とメモ(学内)/予定:学内21/学事予定21/オープンキャンパス21秋/出品(久保)/fig/対称移動1.tex 
%%% Generator=対称移動1.cdy 
{\unitlength=8mm%
\begin{picture}%
(7.45,7.45)(-2.45,-2.45)%
\special{pn 8}%
%
\special{pa  -772   772}\special{pa  -756   756}\special{pa  -693   693}\special{pa  -630   630}%
\special{pa  -567   567}\special{pa  -504   504}\special{pa  -441   441}\special{pa  -378   378}%
\special{pa  -315   315}\special{pa  -252   252}\special{pa  -189   189}\special{pa  -126   126}%
\special{pa   -63    63}\special{pa     0    -0}\special{pa    63   -63}\special{pa   126  -126}%
\special{pa   189  -189}\special{pa   252  -252}\special{pa   315  -315}\special{pa   378  -378}%
\special{pa   441  -441}\special{pa   504  -504}\special{pa   567  -567}\special{pa   630  -630}%
\special{pa   693  -693}\special{pa   756  -756}\special{pa   819  -819}\special{pa   882  -882}%
\special{pa   945  -945}\special{pa  1008 -1008}\special{pa  1071 -1071}\special{pa  1134 -1134}%
\special{pa  1197 -1197}\special{pa  1260 -1260}\special{pa  1323 -1323}\special{pa  1386 -1386}%
\special{pa  1449 -1449}\special{pa  1512 -1512}\special{pa  1575 -1575}%
\special{fp}%
\special{pa 951 -315}\special{pa 951 -316}\special{pa 951 -316}\special{pa 950 -317}%
\special{pa 950 -318}\special{pa 950 -318}\special{pa 949 -319}\special{pa 949 -320}%
\special{pa 948 -320}\special{pa 947 -320}\special{pa 947 -321}\special{pa 946 -321}%
\special{pa 945 -321}\special{pa 945 -321}\special{pa 944 -321}\special{pa 943 -321}%
\special{pa 942 -320}\special{pa 942 -320}\special{pa 941 -320}\special{pa 941 -319}%
\special{pa 940 -318}\special{pa 940 -318}\special{pa 939 -317}\special{pa 939 -316}%
\special{pa 939 -316}\special{pa 939 -315}\special{pa 939 -314}\special{pa 939 -313}%
\special{pa 939 -313}\special{pa 940 -312}\special{pa 940 -311}\special{pa 941 -311}%
\special{pa 941 -310}\special{pa 942 -310}\special{pa 942 -310}\special{pa 943 -309}%
\special{pa 944 -309}\special{pa 945 -309}\special{pa 945 -309}\special{pa 946 -309}%
\special{pa 947 -309}\special{pa 947 -310}\special{pa 948 -310}\special{pa 949 -310}%
\special{pa 949 -311}\special{pa 950 -311}\special{pa 950 -312}\special{pa 950 -313}%
\special{pa 951 -313}\special{pa 951 -314}\special{pa 951 -315}\special{pa 951 -315}%
\special{sh 1}\special{ip}%
\special{pa   951  -315}\special{pa   951  -316}\special{pa   951  -316}\special{pa   950  -317}%
\special{pa   950  -318}\special{pa   950  -318}\special{pa   949  -319}\special{pa   949  -320}%
\special{pa   948  -320}\special{pa   947  -320}\special{pa   947  -321}\special{pa   946  -321}%
\special{pa   945  -321}\special{pa   945  -321}\special{pa   944  -321}\special{pa   943  -321}%
\special{pa   942  -320}\special{pa   942  -320}\special{pa   941  -320}\special{pa   941  -319}%
\special{pa   940  -318}\special{pa   940  -318}\special{pa   939  -317}\special{pa   939  -316}%
\special{pa   939  -316}\special{pa   939  -315}\special{pa   939  -314}\special{pa   939  -313}%
\special{pa   939  -313}\special{pa   940  -312}\special{pa   940  -311}\special{pa   941  -311}%
\special{pa   941  -310}\special{pa   942  -310}\special{pa   942  -310}\special{pa   943  -309}%
\special{pa   944  -309}\special{pa   945  -309}\special{pa   945  -309}\special{pa   946  -309}%
\special{pa   947  -309}\special{pa   947  -310}\special{pa   948  -310}\special{pa   949  -310}%
\special{pa   949  -311}\special{pa   950  -311}\special{pa   950  -312}\special{pa   950  -313}%
\special{pa   951  -313}\special{pa   951  -314}\special{pa   951  -315}%
\special{fp}%
\special{pa 321 -945}\special{pa 321 -946}\special{pa 321 -946}\special{pa 321 -947}%
\special{pa 320 -948}\special{pa 320 -948}\special{pa 319 -949}\special{pa 319 -949}%
\special{pa 318 -950}\special{pa 318 -950}\special{pa 317 -951}\special{pa 316 -951}%
\special{pa 315 -951}\special{pa 315 -951}\special{pa 314 -951}\special{pa 313 -951}%
\special{pa 312 -950}\special{pa 312 -950}\special{pa 311 -949}\special{pa 311 -949}%
\special{pa 310 -948}\special{pa 310 -948}\special{pa 309 -947}\special{pa 309 -946}%
\special{pa 309 -946}\special{pa 309 -945}\special{pa 309 -944}\special{pa 309 -943}%
\special{pa 309 -943}\special{pa 310 -942}\special{pa 310 -941}\special{pa 311 -941}%
\special{pa 311 -940}\special{pa 312 -940}\special{pa 312 -939}\special{pa 313 -939}%
\special{pa 314 -939}\special{pa 315 -939}\special{pa 315 -939}\special{pa 316 -939}%
\special{pa 317 -939}\special{pa 318 -939}\special{pa 318 -940}\special{pa 319 -940}%
\special{pa 319 -941}\special{pa 320 -941}\special{pa 320 -942}\special{pa 321 -943}%
\special{pa 321 -943}\special{pa 321 -944}\special{pa 321 -945}\special{pa 321 -945}%
\special{sh 1}\special{ip}%
\special{pa   321  -945}\special{pa   321  -946}\special{pa   321  -946}\special{pa   321  -947}%
\special{pa   320  -948}\special{pa   320  -948}\special{pa   319  -949}\special{pa   319  -949}%
\special{pa   318  -950}\special{pa   318  -950}\special{pa   317  -951}\special{pa   316  -951}%
\special{pa   315  -951}\special{pa   315  -951}\special{pa   314  -951}\special{pa   313  -951}%
\special{pa   312  -950}\special{pa   312  -950}\special{pa   311  -949}\special{pa   311  -949}%
\special{pa   310  -948}\special{pa   310  -948}\special{pa   309  -947}\special{pa   309  -946}%
\special{pa   309  -946}\special{pa   309  -945}\special{pa   309  -944}\special{pa   309  -943}%
\special{pa   309  -943}\special{pa   310  -942}\special{pa   310  -941}\special{pa   311  -941}%
\special{pa   311  -940}\special{pa   312  -940}\special{pa   312  -939}\special{pa   313  -939}%
\special{pa   314  -939}\special{pa   315  -939}\special{pa   315  -939}\special{pa   316  -939}%
\special{pa   317  -939}\special{pa   318  -939}\special{pa   318  -940}\special{pa   319  -940}%
\special{pa   319  -941}\special{pa   320  -941}\special{pa   320  -942}\special{pa   321  -943}%
\special{pa   321  -943}\special{pa   321  -944}\special{pa   321  -945}%
\special{fp}%
\special{pa   945  -315}\special{pa   315  -945}%
\special{fp}%
\special{pa   776  -450}\special{pa   810  -484}%
\special{fp}%
\special{pa   765  -461}\special{pa   799  -495}%
\special{fp}%
\special{pa   484  -810}\special{pa   450  -776}%
\special{fp}%
\special{pa   495  -799}\special{pa   461  -765}%
\special{fp}%
\special{pa  -315   -20}\special{pa  -315    20}%
\special{fp}%
\settowidth{\Width}{$1$}\setlength{\Width}{-0.5\Width}%
\settoheight{\Height}{$1$}\settodepth{\Depth}{$1$}\setlength{\Height}{-\Height}%
\put(1.0000000,-0.1250000){\hspace*{\Width}\raisebox{\Height}{$1$}}%
%
\special{pa  -630   -20}\special{pa  -630    20}%
\special{fp}%
\settowidth{\Width}{$2$}\setlength{\Width}{-0.5\Width}%
\settoheight{\Height}{$2$}\settodepth{\Depth}{$2$}\setlength{\Height}{-\Height}%
\put(2.0000000,-0.1250000){\hspace*{\Width}\raisebox{\Height}{$2$}}%
%
\special{pa   945   -20}\special{pa   945    20}%
\special{fp}%
\settowidth{\Width}{$3$}\setlength{\Width}{-0.5\Width}%
\settoheight{\Height}{$3$}\settodepth{\Depth}{$3$}\setlength{\Height}{-\Height}%
\put(3.0000000,-0.1250000){\hspace*{\Width}\raisebox{\Height}{$3$}}%
%
\special{pa  1260   -20}\special{pa  1260    20}%
\special{fp}%
\settowidth{\Width}{$4$}\setlength{\Width}{-0.5\Width}%
\settoheight{\Height}{$4$}\settodepth{\Depth}{$4$}\setlength{\Height}{-\Height}%
\put(4.0000000,-0.1250000){\hspace*{\Width}\raisebox{\Height}{$4$}}%
%
\special{pa  -315   -20}\special{pa  -315    20}%
\special{fp}%
\settowidth{\Width}{$-1$}\setlength{\Width}{-0.5\Width}%
\settoheight{\Height}{$-1$}\settodepth{\Depth}{$-1$}\setlength{\Height}{-\Height}%
\put(-1.0000000,-0.1250000){\hspace*{\Width}\raisebox{\Height}{$-1$}}%
%
\special{pa  -630   -20}\special{pa  -630    20}%
\special{fp}%
\settowidth{\Width}{$-2$}\setlength{\Width}{-0.5\Width}%
\settoheight{\Height}{$-2$}\settodepth{\Depth}{$-2$}\setlength{\Height}{-\Height}%
\put(-2.0000000,-0.1250000){\hspace*{\Width}\raisebox{\Height}{$-2$}}%
%
\special{pa    20   315}\special{pa   -20   315}%
\special{fp}%
\settowidth{\Width}{$1$}\setlength{\Width}{-1\Width}%
\settoheight{\Height}{$1$}\settodepth{\Depth}{$1$}\setlength{\Height}{-0.5\Height}\setlength{\Depth}{0.5\Depth}\addtolength{\Height}{\Depth}%
\put(-0.1250000,1.0000000){\hspace*{\Width}\raisebox{\Height}{$1$}}%
%
\special{pa    20   630}\special{pa   -20   630}%
\special{fp}%
\settowidth{\Width}{$2$}\setlength{\Width}{-1\Width}%
\settoheight{\Height}{$2$}\settodepth{\Depth}{$2$}\setlength{\Height}{-0.5\Height}\setlength{\Depth}{0.5\Depth}\addtolength{\Height}{\Depth}%
\put(-0.1250000,2.0000000){\hspace*{\Width}\raisebox{\Height}{$2$}}%
%
\special{pa    20  -945}\special{pa   -20  -945}%
\special{fp}%
\settowidth{\Width}{$3$}\setlength{\Width}{-1\Width}%
\settoheight{\Height}{$3$}\settodepth{\Depth}{$3$}\setlength{\Height}{-0.5\Height}\setlength{\Depth}{0.5\Depth}\addtolength{\Height}{\Depth}%
\put(-0.1250000,3.0000000){\hspace*{\Width}\raisebox{\Height}{$3$}}%
%
\special{pa    20 -1260}\special{pa   -20 -1260}%
\special{fp}%
\settowidth{\Width}{$4$}\setlength{\Width}{-1\Width}%
\settoheight{\Height}{$4$}\settodepth{\Depth}{$4$}\setlength{\Height}{-0.5\Height}\setlength{\Depth}{0.5\Depth}\addtolength{\Height}{\Depth}%
\put(-0.1250000,4.0000000){\hspace*{\Width}\raisebox{\Height}{$4$}}%
%
\special{pa    20   315}\special{pa   -20   315}%
\special{fp}%
\settowidth{\Width}{$-1$}\setlength{\Width}{-1\Width}%
\settoheight{\Height}{$-1$}\settodepth{\Depth}{$-1$}\setlength{\Height}{-0.5\Height}\setlength{\Depth}{0.5\Depth}\addtolength{\Height}{\Depth}%
\put(-0.1250000,-1.0000000){\hspace*{\Width}\raisebox{\Height}{$-1$}}%
%
\special{pa    20   630}\special{pa   -20   630}%
\special{fp}%
\settowidth{\Width}{$-2$}\setlength{\Width}{-1\Width}%
\settoheight{\Height}{$-2$}\settodepth{\Depth}{$-2$}\setlength{\Height}{-0.5\Height}\setlength{\Depth}{0.5\Depth}\addtolength{\Height}{\Depth}%
\put(-0.1250000,-2.0000000){\hspace*{\Width}\raisebox{\Height}{$-2$}}%
%
\settowidth{\Width}{P}\setlength{\Width}{0\Width}%
\settoheight{\Height}{P}\settodepth{\Depth}{P}\setlength{\Height}{-0.5\Height}\setlength{\Depth}{0.5\Depth}\addtolength{\Height}{\Depth}%
\put(3.1875000,1.0000000){\hspace*{\Width}\raisebox{\Height}{P}}%
%
\settowidth{\Width}{Q}\setlength{\Width}{0\Width}%
\settoheight{\Height}{Q}\settodepth{\Depth}{Q}\setlength{\Height}{\Depth}%
\put(1.0625000,3.1875000){\hspace*{\Width}\raisebox{\Height}{Q}}%
%
\special{pa 1500 24}\special{pa 1575 0}\special{pa 1500 -24}\special{pa 1515 0}\special{pa 1500 24}%
\special{pa 1500 24}\special{sh 1}\special{ip}%
\special{pn 1}%
\special{pa  1500    24}\special{pa  1575    -0}\special{pa  1500   -24}\special{pa  1515    -0}%
\special{pa  1500    24}%
\special{fp}%
\special{pn 8}%
\special{pn 8}%
\special{pa  -772    -0}\special{pa  1515    -0}%
\special{fp}%
\special{pn 8}%
\special{pa 24 -1500}\special{pa 0 -1575}\special{pa -24 -1500}\special{pa 0 -1515}%
\special{pa 24 -1500}\special{pa 24 -1500}\special{sh 1}\special{ip}%
\special{pn 1}%
\special{pa    24 -1500}\special{pa     0 -1575}\special{pa   -24 -1500}\special{pa     0 -1515}%
\special{pa    24 -1500}%
\special{fp}%
\special{pn 8}%
\special{pn 8}%
\special{pa     0   772}\special{pa     0 -1515}%
\special{fp}%
\special{pn 8}%
\settowidth{\Width}{$x$}\setlength{\Width}{0\Width}%
\settoheight{\Height}{$x$}\settodepth{\Depth}{$x$}\setlength{\Height}{-0.5\Height}\setlength{\Depth}{0.5\Depth}\addtolength{\Height}{\Depth}%
\put(5.0625000,0.0000000){\hspace*{\Width}\raisebox{\Height}{$x$}}%
%
\settowidth{\Width}{$y$}\setlength{\Width}{-0.5\Width}%
\settoheight{\Height}{$y$}\settodepth{\Depth}{$y$}\setlength{\Height}{\Depth}%
\put(0.0000000,5.0625000){\hspace*{\Width}\raisebox{\Height}{$y$}}%
%
\settowidth{\Width}{O}\setlength{\Width}{0\Width}%
\settoheight{\Height}{O}\settodepth{\Depth}{O}\setlength{\Height}{-\Height}%
\put(0.0625000,-0.0625000){\hspace*{\Width}\raisebox{\Height}{O}}%
%
\end{picture}}%}
\end{layer}\par
\vspace{15\baselineskip}

下のQRコードからリンクへ飛んでみて。\\
点Pをつまんで動かすと,点Qも動くよ。\\

%% upLaTeXでは,図の挿入にBBdataが必須
\includegraphics[bb=18 18 170 170,width=50mm,clip]{対称移動1QR.png}\\

\url{https://kubo-yuge.github.io/class2019/taisyoidou1/taisyoidou1json.html}\\

\columnbreak

\subsection{別の直線でも}

直線をはさむ2点を見てみよう。\par
\begin{layer}{80}{0}
\putnotese{7}{7}{\input{fig/対称移動2.tex}}
\end{layer}\par
\vspace{15\baselineskip}

下のQRコードからリンクへ飛んでみて。\\
点Pをつまんで動かすと,点Qも動くよ。\\

%% upLaTeXでは,図の挿入にBBdataが必須
\includegraphics[bb=18 18 170 170,width=50mm,clip]{対称移動2QR.png}\\

\url{https://kubo-yuge.github.io/class2019/taisyoidou2/taisyoidou2json.html}\\

\columnbreak

\subsection{点の座標を分数で}

直線をはさむ2点の座標に注目!!\par
\begin{layer}{80}{0}
\putnotese{3}{7}{%%% /Users/kumac21/DataKubo_Mac21/予定とメモなど/予定とメモ(学内)/予定:学内21/学事予定21/オープンキャンパス21秋/出品(久保)/fig/対称移動3.tex 
%%% Generator=対称移動3.cdy 
{\unitlength=6mm%
\begin{picture}%
(10,10)(-5,-5)%
\special{pn 8}%
%
{%
\color[cmyk]{1,1,0,0}%
\special{pa  -591  1181}\special{pa   591 -1181}%
\special{fp}%
}%
\special{pa  -236   -20}\special{pa  -236    20}%
\special{fp}%
\settowidth{\Width}{$1$}\setlength{\Width}{-0.5\Width}%
\settoheight{\Height}{$1$}\settodepth{\Depth}{$1$}\setlength{\Height}{-\Height}%
\put(1.0000000,-0.1666667){\hspace*{\Width}\raisebox{\Height}{$1$}}%
%
\special{pa  -472   -20}\special{pa  -472    20}%
\special{fp}%
\settowidth{\Width}{$2$}\setlength{\Width}{-0.5\Width}%
\settoheight{\Height}{$2$}\settodepth{\Depth}{$2$}\setlength{\Height}{-\Height}%
\put(2.0000000,-0.1666667){\hspace*{\Width}\raisebox{\Height}{$2$}}%
%
\special{pa  -709   -20}\special{pa  -709    20}%
\special{fp}%
\settowidth{\Width}{$3$}\setlength{\Width}{-0.5\Width}%
\settoheight{\Height}{$3$}\settodepth{\Depth}{$3$}\setlength{\Height}{-\Height}%
\put(3.0000000,-0.1666667){\hspace*{\Width}\raisebox{\Height}{$3$}}%
%
\special{pa  -945   -20}\special{pa  -945    20}%
\special{fp}%
\settowidth{\Width}{$4$}\setlength{\Width}{-0.5\Width}%
\settoheight{\Height}{$4$}\settodepth{\Depth}{$4$}\setlength{\Height}{-\Height}%
\put(4.0000000,-0.1666667){\hspace*{\Width}\raisebox{\Height}{$4$}}%
%
\special{pa  -236   -20}\special{pa  -236    20}%
\special{fp}%
\settowidth{\Width}{$-1$}\setlength{\Width}{-0.5\Width}%
\settoheight{\Height}{$-1$}\settodepth{\Depth}{$-1$}\setlength{\Height}{-\Height}%
\put(-1.0000000,-0.1666667){\hspace*{\Width}\raisebox{\Height}{$-1$}}%
%
\special{pa  -472   -20}\special{pa  -472    20}%
\special{fp}%
\settowidth{\Width}{$-2$}\setlength{\Width}{-0.5\Width}%
\settoheight{\Height}{$-2$}\settodepth{\Depth}{$-2$}\setlength{\Height}{-\Height}%
\put(-2.0000000,-0.1666667){\hspace*{\Width}\raisebox{\Height}{$-2$}}%
%
\special{pa  -709   -20}\special{pa  -709    20}%
\special{fp}%
\settowidth{\Width}{$-3$}\setlength{\Width}{-0.5\Width}%
\settoheight{\Height}{$-3$}\settodepth{\Depth}{$-3$}\setlength{\Height}{-\Height}%
\put(-3.0000000,-0.1666667){\hspace*{\Width}\raisebox{\Height}{$-3$}}%
%
\special{pa  -945   -20}\special{pa  -945    20}%
\special{fp}%
\settowidth{\Width}{$-4$}\setlength{\Width}{-0.5\Width}%
\settoheight{\Height}{$-4$}\settodepth{\Depth}{$-4$}\setlength{\Height}{-\Height}%
\put(-4.0000000,-0.1666667){\hspace*{\Width}\raisebox{\Height}{$-4$}}%
%
\special{pa    20   236}\special{pa   -20   236}%
\special{fp}%
\settowidth{\Width}{$1$}\setlength{\Width}{-1\Width}%
\settoheight{\Height}{$1$}\settodepth{\Depth}{$1$}\setlength{\Height}{-0.5\Height}\setlength{\Depth}{0.5\Depth}\addtolength{\Height}{\Depth}%
\put(-0.1666667,1.0000000){\hspace*{\Width}\raisebox{\Height}{$1$}}%
%
\special{pa    20   472}\special{pa   -20   472}%
\special{fp}%
\settowidth{\Width}{$2$}\setlength{\Width}{-1\Width}%
\settoheight{\Height}{$2$}\settodepth{\Depth}{$2$}\setlength{\Height}{-0.5\Height}\setlength{\Depth}{0.5\Depth}\addtolength{\Height}{\Depth}%
\put(-0.1666667,2.0000000){\hspace*{\Width}\raisebox{\Height}{$2$}}%
%
\special{pa    20   709}\special{pa   -20   709}%
\special{fp}%
\settowidth{\Width}{$3$}\setlength{\Width}{-1\Width}%
\settoheight{\Height}{$3$}\settodepth{\Depth}{$3$}\setlength{\Height}{-0.5\Height}\setlength{\Depth}{0.5\Depth}\addtolength{\Height}{\Depth}%
\put(-0.1666667,3.0000000){\hspace*{\Width}\raisebox{\Height}{$3$}}%
%
\special{pa    20   945}\special{pa   -20   945}%
\special{fp}%
\settowidth{\Width}{$4$}\setlength{\Width}{-1\Width}%
\settoheight{\Height}{$4$}\settodepth{\Depth}{$4$}\setlength{\Height}{-0.5\Height}\setlength{\Depth}{0.5\Depth}\addtolength{\Height}{\Depth}%
\put(-0.1666667,4.0000000){\hspace*{\Width}\raisebox{\Height}{$4$}}%
%
\special{pa    20   236}\special{pa   -20   236}%
\special{fp}%
\settowidth{\Width}{$-1$}\setlength{\Width}{-1\Width}%
\settoheight{\Height}{$-1$}\settodepth{\Depth}{$-1$}\setlength{\Height}{-0.5\Height}\setlength{\Depth}{0.5\Depth}\addtolength{\Height}{\Depth}%
\put(-0.1666667,-1.0000000){\hspace*{\Width}\raisebox{\Height}{$-1$}}%
%
\special{pa    20   472}\special{pa   -20   472}%
\special{fp}%
\settowidth{\Width}{$-2$}\setlength{\Width}{-1\Width}%
\settoheight{\Height}{$-2$}\settodepth{\Depth}{$-2$}\setlength{\Height}{-0.5\Height}\setlength{\Depth}{0.5\Depth}\addtolength{\Height}{\Depth}%
\put(-0.1666667,-2.0000000){\hspace*{\Width}\raisebox{\Height}{$-2$}}%
%
\special{pa    20   709}\special{pa   -20   709}%
\special{fp}%
\settowidth{\Width}{$-3$}\setlength{\Width}{-1\Width}%
\settoheight{\Height}{$-3$}\settodepth{\Depth}{$-3$}\setlength{\Height}{-0.5\Height}\setlength{\Depth}{0.5\Depth}\addtolength{\Height}{\Depth}%
\put(-0.1666667,-3.0000000){\hspace*{\Width}\raisebox{\Height}{$-3$}}%
%
\special{pa    20   945}\special{pa   -20   945}%
\special{fp}%
\settowidth{\Width}{$-4$}\setlength{\Width}{-1\Width}%
\settoheight{\Height}{$-4$}\settodepth{\Depth}{$-4$}\setlength{\Height}{-0.5\Height}\setlength{\Depth}{0.5\Depth}\addtolength{\Height}{\Depth}%
\put(-0.1666667,-4.0000000){\hspace*{\Width}\raisebox{\Height}{$-4$}}%
%
\special{pa -204 -841}\special{pa -204 -843}\special{pa -204 -844}\special{pa -205 -846}%
\special{pa -206 -847}\special{pa -206 -849}\special{pa -208 -850}\special{pa -209 -851}%
\special{pa -210 -852}\special{pa -212 -853}\special{pa -213 -854}\special{pa -215 -854}%
\special{pa -216 -854}\special{pa -218 -854}\special{pa -220 -854}\special{pa -221 -854}%
\special{pa -223 -853}\special{pa -225 -852}\special{pa -226 -851}\special{pa -227 -850}%
\special{pa -228 -849}\special{pa -229 -847}\special{pa -230 -846}\special{pa -230 -844}%
\special{pa -231 -843}\special{pa -231 -841}\special{pa -231 -839}\special{pa -230 -838}%
\special{pa -230 -836}\special{pa -229 -834}\special{pa -228 -833}\special{pa -227 -832}%
\special{pa -226 -831}\special{pa -225 -830}\special{pa -223 -829}\special{pa -221 -828}%
\special{pa -220 -828}\special{pa -218 -828}\special{pa -216 -828}\special{pa -215 -828}%
\special{pa -213 -828}\special{pa -212 -829}\special{pa -210 -830}\special{pa -209 -831}%
\special{pa -208 -832}\special{pa -206 -833}\special{pa -206 -834}\special{pa -205 -836}%
\special{pa -204 -838}\special{pa -204 -839}\special{pa -204 -841}\special{pa -204 -841}%
\special{sh 1}\special{ip}%
\special{pa  -204  -841}\special{pa  -204  -843}\special{pa  -204  -844}\special{pa  -205  -846}%
\special{pa  -206  -847}\special{pa  -206  -849}\special{pa  -208  -850}\special{pa  -209  -851}%
\special{pa  -210  -852}\special{pa  -212  -853}\special{pa  -213  -854}\special{pa  -215  -854}%
\special{pa  -216  -854}\special{pa  -218  -854}\special{pa  -220  -854}\special{pa  -221  -854}%
\special{pa  -223  -853}\special{pa  -225  -852}\special{pa  -226  -851}\special{pa  -227  -850}%
\special{pa  -228  -849}\special{pa  -229  -847}\special{pa  -230  -846}\special{pa  -230  -844}%
\special{pa  -231  -843}\special{pa  -231  -841}\special{pa  -231  -839}\special{pa  -230  -838}%
\special{pa  -230  -836}\special{pa  -229  -834}\special{pa  -228  -833}\special{pa  -227  -832}%
\special{pa  -226  -831}\special{pa  -225  -830}\special{pa  -223  -829}\special{pa  -221  -828}%
\special{pa  -220  -828}\special{pa  -218  -828}\special{pa  -216  -828}\special{pa  -215  -828}%
\special{pa  -213  -828}\special{pa  -212  -829}\special{pa  -210  -830}\special{pa  -209  -831}%
\special{pa  -208  -832}\special{pa  -206  -833}\special{pa  -206  -834}\special{pa  -205  -836}%
\special{pa  -204  -838}\special{pa  -204  -839}\special{pa  -204  -841}%
\special{fp}%
\special{pa   803  -331}\special{pa  -217  -841}%
\special{fp}%
{%
\color[cmyk]{1,1,0,0}%
\special{pa 306 -586}\special{pa 306 -588}\special{pa 306 -589}\special{pa 305 -591}%
\special{pa 305 -592}\special{pa 304 -594}\special{pa 303 -595}\special{pa 301 -596}%
\special{pa 300 -597}\special{pa 299 -598}\special{pa 297 -599}\special{pa 295 -599}%
\special{pa 294 -599}\special{pa 292 -599}\special{pa 290 -599}\special{pa 289 -599}%
\special{pa 287 -598}\special{pa 286 -597}\special{pa 284 -596}\special{pa 283 -595}%
\special{pa 282 -594}\special{pa 281 -592}\special{pa 280 -591}\special{pa 280 -589}%
\special{pa 280 -588}\special{pa 279 -586}\special{pa 280 -584}\special{pa 280 -582}%
\special{pa 280 -581}\special{pa 281 -579}\special{pa 282 -578}\special{pa 283 -577}%
\special{pa 284 -575}\special{pa 286 -574}\special{pa 287 -574}\special{pa 289 -573}%
\special{pa 290 -573}\special{pa 292 -572}\special{pa 294 -572}\special{pa 295 -573}%
\special{pa 297 -573}\special{pa 299 -574}\special{pa 300 -574}\special{pa 301 -575}%
\special{pa 303 -577}\special{pa 304 -578}\special{pa 305 -579}\special{pa 305 -581}%
\special{pa 306 -582}\special{pa 306 -584}\special{pa 306 -586}\special{pa 306 -586}%
\special{sh 1}\special{ip}%
}%
\special{pa   306  -586}\special{pa   306  -588}\special{pa   306  -589}\special{pa   305  -591}%
\special{pa   305  -592}\special{pa   304  -594}\special{pa   303  -595}\special{pa   301  -596}%
\special{pa   300  -597}\special{pa   299  -598}\special{pa   297  -599}\special{pa   295  -599}%
\special{pa   294  -599}\special{pa   292  -599}\special{pa   290  -599}\special{pa   289  -599}%
\special{pa   287  -598}\special{pa   286  -597}\special{pa   284  -596}\special{pa   283  -595}%
\special{pa   282  -594}\special{pa   281  -592}\special{pa   280  -591}\special{pa   280  -589}%
\special{pa   280  -588}\special{pa   279  -586}\special{pa   280  -584}\special{pa   280  -582}%
\special{pa   280  -581}\special{pa   281  -579}\special{pa   282  -578}\special{pa   283  -577}%
\special{pa   284  -575}\special{pa   286  -574}\special{pa   287  -574}\special{pa   289  -573}%
\special{pa   290  -573}\special{pa   292  -572}\special{pa   294  -572}\special{pa   295  -573}%
\special{pa   297  -573}\special{pa   299  -574}\special{pa   300  -574}\special{pa   301  -575}%
\special{pa   303  -577}\special{pa   304  -578}\special{pa   305  -579}\special{pa   305  -581}%
\special{pa   306  -582}\special{pa   306  -584}\special{pa   306  -586}%
\special{fp}%
\special{pa   545  -440}\special{pa   561  -471}%
\special{fp}%
\special{pa   535  -445}\special{pa   551  -477}%
\special{fp}%
\special{pa    40  -732}\special{pa    25  -700}%
\special{fp}%
\special{pa    51  -727}\special{pa    35  -695}%
\special{fp}%
\settowidth{\Width}{$\mathrm{P}\displaystyle(\frac{17}{5},\frac{7}{5})$}\setlength{\Width}{0\Width}%
\settoheight{\Height}{$\mathrm{P}\displaystyle(\frac{17}{5},\frac{7}{5})$}\settodepth{\Depth}{$\mathrm{P}\displaystyle(\frac{17}{5},\frac{7}{5})$}\setlength{\Height}{\Depth}%
\put(3.6500000,1.6500000){\hspace*{\Width}\raisebox{\Height}{$\mathrm{P}\displaystyle(\frac{17}{5},\frac{7}{5})$}}%
%
\settowidth{\Width}{$\mathrm{Q}\displaystyle(\frac{-23}{25},\frac{89}{25})$}\setlength{\Width}{-1\Width}%
\settoheight{\Height}{$\mathrm{Q}\displaystyle(\frac{-23}{25},\frac{89}{25})$}\settodepth{\Depth}{$\mathrm{Q}\displaystyle(\frac{-23}{25},\frac{89}{25})$}\setlength{\Height}{\Depth}%
\put(-1.1700000,3.8100000){\hspace*{\Width}\raisebox{\Height}{$\mathrm{Q}\displaystyle(\frac{-23}{25},\frac{89}{25})$}}%
%
\settowidth{\Width}{$y=2x$}\setlength{\Width}{0\Width}%
\settoheight{\Height}{$y=2x$}\settodepth{\Depth}{$y=2x$}\setlength{\Height}{-0.5\Height}\setlength{\Depth}{0.5\Depth}\addtolength{\Height}{\Depth}%
\put(3.0833333,5.0000000){\hspace*{\Width}\raisebox{\Height}{$y=2x$}}%
%
\special{pa 1106 24}\special{pa 1181 0}\special{pa 1106 -24}\special{pa 1121 0}\special{pa 1106 24}%
\special{pa 1106 24}\special{sh 1}\special{ip}%
\special{pn 1}%
\special{pa  1106    24}\special{pa  1181    -0}\special{pa  1106   -24}\special{pa  1121    -0}%
\special{pa  1106    24}%
\special{fp}%
\special{pn 8}%
\special{pn 8}%
\special{pa -1181    -0}\special{pa  1121    -0}%
\special{fp}%
\special{pn 8}%
\special{pa 24 -1106}\special{pa 0 -1181}\special{pa -24 -1106}\special{pa 0 -1121}%
\special{pa 24 -1106}\special{pa 24 -1106}\special{sh 1}\special{ip}%
\special{pn 1}%
\special{pa    24 -1106}\special{pa     0 -1181}\special{pa   -24 -1106}\special{pa     0 -1121}%
\special{pa    24 -1106}%
\special{fp}%
\special{pn 8}%
\special{pn 8}%
\special{pa     0  1181}\special{pa     0 -1121}%
\special{fp}%
\special{pn 8}%
\settowidth{\Width}{$x$}\setlength{\Width}{0\Width}%
\settoheight{\Height}{$x$}\settodepth{\Depth}{$x$}\setlength{\Height}{-0.5\Height}\setlength{\Depth}{0.5\Depth}\addtolength{\Height}{\Depth}%
\put(5.0833333,0.0000000){\hspace*{\Width}\raisebox{\Height}{$x$}}%
%
\settowidth{\Width}{$y$}\setlength{\Width}{-0.5\Width}%
\settoheight{\Height}{$y$}\settodepth{\Depth}{$y$}\setlength{\Height}{\Depth}%
\put(0.0000000,5.0833333){\hspace*{\Width}\raisebox{\Height}{$y$}}%
%
\settowidth{\Width}{O}\setlength{\Width}{0\Width}%
\settoheight{\Height}{O}\settodepth{\Depth}{O}\setlength{\Height}{-\Height}%
\put(0.0833333,-0.0833333){\hspace*{\Width}\raisebox{\Height}{O}}%
%
\end{picture}}%}
\end{layer}\par
\vspace{15\baselineskip}

下のQRコードからリンクへ飛んでみて。\\
点Pをつまんで動かすと,点Qも動くよ。\\

%% upLaTeXでは,図の挿入にBBdataが必須
\includegraphics[bb=18 18 185 185,width=50mm,clip]{対称移動3QR.png}\\

\url{https://kubo-yuge.github.io/class2019/taisyoidou2frac3/taisyoidou2_frac3json.html}\\

\columnbreak

\subsection{直線の傾きを変えてみよう}

白い枠の中の$m=$のあとに,整数や分数を入れると,直線の傾きが変わるよ。負の数でもいけるよ。\par
$1,\ -1,\ 2/3,\ -3/2$などでお願いします。\par
\begin{layer}{80}{0}
\putnotese{3}{7}{%%% /Users/kumac21/DataKubo_Mac21/予定とメモなど/予定とメモ(学内)/予定:学内21/学事予定21/オープンキャンパス21秋/出品(久保)/fig/対称移動4.tex 
%%% Generator=対称移動4.cdy 
{\unitlength=6mm%
\begin{picture}%
(10,10)(-5,-5)%
\special{pn 8}%
%
\settowidth{\Width}{$m=2$}\setlength{\Width}{-1\Width}%
\settoheight{\Height}{$m=2$}\settodepth{\Depth}{$m=2$}\setlength{\Height}{\Depth}%
\put(4.9166667,4.0833333){\hspace*{\Width}\raisebox{\Height}{$m=2$}}%
%
\special{pa  -591  1181}\special{pa   591 -1181}%
\special{fp}%
\special{pa  -945   -20}\special{pa  -945    20}%
\special{fp}%
\settowidth{\Width}{$-4$}\setlength{\Width}{-0.5\Width}%
\settoheight{\Height}{$-4$}\settodepth{\Depth}{$-4$}\setlength{\Height}{-\Height}%
\put(-4.0000000,-0.1666667){\hspace*{\Width}\raisebox{\Height}{$-4$}}%
%
\special{pa  -709   -20}\special{pa  -709    20}%
\special{fp}%
\settowidth{\Width}{$-3$}\setlength{\Width}{-0.5\Width}%
\settoheight{\Height}{$-3$}\settodepth{\Depth}{$-3$}\setlength{\Height}{-\Height}%
\put(-3.0000000,-0.1666667){\hspace*{\Width}\raisebox{\Height}{$-3$}}%
%
\special{pa  -472   -20}\special{pa  -472    20}%
\special{fp}%
\settowidth{\Width}{$-2$}\setlength{\Width}{-0.5\Width}%
\settoheight{\Height}{$-2$}\settodepth{\Depth}{$-2$}\setlength{\Height}{-\Height}%
\put(-2.0000000,-0.1666667){\hspace*{\Width}\raisebox{\Height}{$-2$}}%
%
\special{pa  -236   -20}\special{pa  -236    20}%
\special{fp}%
\settowidth{\Width}{$-1$}\setlength{\Width}{-0.5\Width}%
\settoheight{\Height}{$-1$}\settodepth{\Depth}{$-1$}\setlength{\Height}{-\Height}%
\put(-1.0000000,-0.1666667){\hspace*{\Width}\raisebox{\Height}{$-1$}}%
%
\special{pa   236   -20}\special{pa   236    20}%
\special{fp}%
\settowidth{\Width}{$1$}\setlength{\Width}{-0.5\Width}%
\settoheight{\Height}{$1$}\settodepth{\Depth}{$1$}\setlength{\Height}{-\Height}%
\put(1.0000000,-0.1666667){\hspace*{\Width}\raisebox{\Height}{$1$}}%
%
\special{pa   472   -20}\special{pa   472    20}%
\special{fp}%
\settowidth{\Width}{$2$}\setlength{\Width}{-0.5\Width}%
\settoheight{\Height}{$2$}\settodepth{\Depth}{$2$}\setlength{\Height}{-\Height}%
\put(2.0000000,-0.1666667){\hspace*{\Width}\raisebox{\Height}{$2$}}%
%
\special{pa   709   -20}\special{pa   709    20}%
\special{fp}%
\settowidth{\Width}{$3$}\setlength{\Width}{-0.5\Width}%
\settoheight{\Height}{$3$}\settodepth{\Depth}{$3$}\setlength{\Height}{-\Height}%
\put(3.0000000,-0.1666667){\hspace*{\Width}\raisebox{\Height}{$3$}}%
%
\special{pa   945   -20}\special{pa   945    20}%
\special{fp}%
\settowidth{\Width}{$4$}\setlength{\Width}{-0.5\Width}%
\settoheight{\Height}{$4$}\settodepth{\Depth}{$4$}\setlength{\Height}{-\Height}%
\put(4.0000000,-0.1666667){\hspace*{\Width}\raisebox{\Height}{$4$}}%
%
\special{pa    20   945}\special{pa   -20   945}%
\special{fp}%
\settowidth{\Width}{$-4$}\setlength{\Width}{-1\Width}%
\settoheight{\Height}{$-4$}\settodepth{\Depth}{$-4$}\setlength{\Height}{-0.5\Height}\setlength{\Depth}{0.5\Depth}\addtolength{\Height}{\Depth}%
\put(-0.1666667,-4.0000000){\hspace*{\Width}\raisebox{\Height}{$-4$}}%
%
\special{pa    20   709}\special{pa   -20   709}%
\special{fp}%
\settowidth{\Width}{$-3$}\setlength{\Width}{-1\Width}%
\settoheight{\Height}{$-3$}\settodepth{\Depth}{$-3$}\setlength{\Height}{-0.5\Height}\setlength{\Depth}{0.5\Depth}\addtolength{\Height}{\Depth}%
\put(-0.1666667,-3.0000000){\hspace*{\Width}\raisebox{\Height}{$-3$}}%
%
\special{pa    20   472}\special{pa   -20   472}%
\special{fp}%
\settowidth{\Width}{$-2$}\setlength{\Width}{-1\Width}%
\settoheight{\Height}{$-2$}\settodepth{\Depth}{$-2$}\setlength{\Height}{-0.5\Height}\setlength{\Depth}{0.5\Depth}\addtolength{\Height}{\Depth}%
\put(-0.1666667,-2.0000000){\hspace*{\Width}\raisebox{\Height}{$-2$}}%
%
\special{pa    20   236}\special{pa   -20   236}%
\special{fp}%
\settowidth{\Width}{$-1$}\setlength{\Width}{-1\Width}%
\settoheight{\Height}{$-1$}\settodepth{\Depth}{$-1$}\setlength{\Height}{-0.5\Height}\setlength{\Depth}{0.5\Depth}\addtolength{\Height}{\Depth}%
\put(-0.1666667,-1.0000000){\hspace*{\Width}\raisebox{\Height}{$-1$}}%
%
\special{pa    20  -236}\special{pa   -20  -236}%
\special{fp}%
\settowidth{\Width}{$1$}\setlength{\Width}{-1\Width}%
\settoheight{\Height}{$1$}\settodepth{\Depth}{$1$}\setlength{\Height}{-0.5\Height}\setlength{\Depth}{0.5\Depth}\addtolength{\Height}{\Depth}%
\put(-0.1666667,1.0000000){\hspace*{\Width}\raisebox{\Height}{$1$}}%
%
\special{pa    20  -472}\special{pa   -20  -472}%
\special{fp}%
\settowidth{\Width}{$2$}\setlength{\Width}{-1\Width}%
\settoheight{\Height}{$2$}\settodepth{\Depth}{$2$}\setlength{\Height}{-0.5\Height}\setlength{\Depth}{0.5\Depth}\addtolength{\Height}{\Depth}%
\put(-0.1666667,2.0000000){\hspace*{\Width}\raisebox{\Height}{$2$}}%
%
\special{pa    20  -709}\special{pa   -20  -709}%
\special{fp}%
\settowidth{\Width}{$3$}\setlength{\Width}{-1\Width}%
\settoheight{\Height}{$3$}\settodepth{\Depth}{$3$}\setlength{\Height}{-0.5\Height}\setlength{\Depth}{0.5\Depth}\addtolength{\Height}{\Depth}%
\put(-0.1666667,3.0000000){\hspace*{\Width}\raisebox{\Height}{$3$}}%
%
\special{pa    20  -945}\special{pa   -20  -945}%
\special{fp}%
\settowidth{\Width}{$4$}\setlength{\Width}{-1\Width}%
\settoheight{\Height}{$4$}\settodepth{\Depth}{$4$}\setlength{\Height}{-0.5\Height}\setlength{\Depth}{0.5\Depth}\addtolength{\Height}{\Depth}%
\put(-0.1666667,4.0000000){\hspace*{\Width}\raisebox{\Height}{$4$}}%
%
\special{pa 486 -236}\special{pa 486 -238}\special{pa 485 -240}\special{pa 485 -241}%
\special{pa 484 -243}\special{pa 483 -244}\special{pa 482 -245}\special{pa 481 -247}%
\special{pa 480 -248}\special{pa 478 -248}\special{pa 477 -249}\special{pa 475 -249}%
\special{pa 473 -250}\special{pa 472 -250}\special{pa 470 -249}\special{pa 468 -249}%
\special{pa 467 -248}\special{pa 465 -248}\special{pa 464 -247}\special{pa 463 -245}%
\special{pa 462 -244}\special{pa 461 -243}\special{pa 460 -241}\special{pa 459 -240}%
\special{pa 459 -238}\special{pa 459 -236}\special{pa 459 -235}\special{pa 459 -233}%
\special{pa 460 -231}\special{pa 461 -230}\special{pa 462 -228}\special{pa 463 -227}%
\special{pa 464 -226}\special{pa 465 -225}\special{pa 467 -224}\special{pa 468 -223}%
\special{pa 470 -223}\special{pa 472 -223}\special{pa 473 -223}\special{pa 475 -223}%
\special{pa 477 -223}\special{pa 478 -224}\special{pa 480 -225}\special{pa 481 -226}%
\special{pa 482 -227}\special{pa 483 -228}\special{pa 484 -230}\special{pa 485 -231}%
\special{pa 485 -233}\special{pa 486 -235}\special{pa 486 -236}\special{pa 486 -236}%
\special{sh 1}\special{ip}%
\special{pa   486  -236}\special{pa   486  -238}\special{pa   485  -240}\special{pa   485  -241}%
\special{pa   484  -243}\special{pa   483  -244}\special{pa   482  -245}\special{pa   481  -247}%
\special{pa   480  -248}\special{pa   478  -248}\special{pa   477  -249}\special{pa   475  -249}%
\special{pa   473  -250}\special{pa   472  -250}\special{pa   470  -249}\special{pa   468  -249}%
\special{pa   467  -248}\special{pa   465  -248}\special{pa   464  -247}\special{pa   463  -245}%
\special{pa   462  -244}\special{pa   461  -243}\special{pa   460  -241}\special{pa   459  -240}%
\special{pa   459  -238}\special{pa   459  -236}\special{pa   459  -235}\special{pa   459  -233}%
\special{pa   460  -231}\special{pa   461  -230}\special{pa   462  -228}\special{pa   463  -227}%
\special{pa   464  -226}\special{pa   465  -225}\special{pa   467  -224}\special{pa   468  -223}%
\special{pa   470  -223}\special{pa   472  -223}\special{pa   473  -223}\special{pa   475  -223}%
\special{pa   477  -223}\special{pa   478  -224}\special{pa   480  -225}\special{pa   481  -226}%
\special{pa   482  -227}\special{pa   483  -228}\special{pa   484  -230}\special{pa   485  -231}%
\special{pa   485  -233}\special{pa   486  -235}\special{pa   486  -236}%
\special{fp}%
\special{pa -81 -520}\special{pa -81 -521}\special{pa -81 -523}\special{pa -82 -525}%
\special{pa -83 -526}\special{pa -84 -528}\special{pa -85 -529}\special{pa -86 -530}%
\special{pa -87 -531}\special{pa -89 -532}\special{pa -90 -532}\special{pa -92 -533}%
\special{pa -94 -533}\special{pa -95 -533}\special{pa -97 -533}\special{pa -99 -532}%
\special{pa -100 -532}\special{pa -102 -531}\special{pa -103 -530}\special{pa -104 -529}%
\special{pa -105 -528}\special{pa -106 -526}\special{pa -107 -525}\special{pa -108 -523}%
\special{pa -108 -521}\special{pa -108 -520}\special{pa -108 -518}\special{pa -108 -516}%
\special{pa -107 -515}\special{pa -106 -513}\special{pa -105 -512}\special{pa -104 -510}%
\special{pa -103 -509}\special{pa -102 -508}\special{pa -100 -508}\special{pa -99 -507}%
\special{pa -97 -506}\special{pa -95 -506}\special{pa -94 -506}\special{pa -92 -506}%
\special{pa -90 -507}\special{pa -89 -508}\special{pa -87 -508}\special{pa -86 -509}%
\special{pa -85 -510}\special{pa -84 -512}\special{pa -83 -513}\special{pa -82 -515}%
\special{pa -81 -516}\special{pa -81 -518}\special{pa -81 -520}\special{pa -81 -520}%
\special{sh 1}\special{ip}%
\special{pa   -81  -520}\special{pa   -81  -521}\special{pa   -81  -523}\special{pa   -82  -525}%
\special{pa   -83  -526}\special{pa   -84  -528}\special{pa   -85  -529}\special{pa   -86  -530}%
\special{pa   -87  -531}\special{pa   -89  -532}\special{pa   -90  -532}\special{pa   -92  -533}%
\special{pa   -94  -533}\special{pa   -95  -533}\special{pa   -97  -533}\special{pa   -99  -532}%
\special{pa  -100  -532}\special{pa  -102  -531}\special{pa  -103  -530}\special{pa  -104  -529}%
\special{pa  -105  -528}\special{pa  -106  -526}\special{pa  -107  -525}\special{pa  -108  -523}%
\special{pa  -108  -521}\special{pa  -108  -520}\special{pa  -108  -518}\special{pa  -108  -516}%
\special{pa  -107  -515}\special{pa  -106  -513}\special{pa  -105  -512}\special{pa  -104  -510}%
\special{pa  -103  -509}\special{pa  -102  -508}\special{pa  -100  -508}\special{pa   -99  -507}%
\special{pa   -97  -506}\special{pa   -95  -506}\special{pa   -94  -506}\special{pa   -92  -506}%
\special{pa   -90  -507}\special{pa   -89  -508}\special{pa   -87  -508}\special{pa   -86  -509}%
\special{pa   -85  -510}\special{pa   -84  -512}\special{pa   -83  -513}\special{pa   -82  -515}%
\special{pa   -81  -516}\special{pa   -81  -518}\special{pa   -81  -520}%
\special{fp}%
\special{pa   472  -236}\special{pa   -94  -520}%
\special{fp}%
{%
\color[cmyk]{1,1,0,0}%
\special{pa 202 -378}\special{pa 202 -380}\special{pa 202 -381}\special{pa 201 -383}%
\special{pa 201 -384}\special{pa 200 -386}\special{pa 199 -387}\special{pa 198 -388}%
\special{pa 196 -389}\special{pa 195 -390}\special{pa 193 -391}\special{pa 191 -391}%
\special{pa 190 -391}\special{pa 188 -391}\special{pa 186 -391}\special{pa 185 -391}%
\special{pa 183 -390}\special{pa 182 -389}\special{pa 180 -388}\special{pa 179 -387}%
\special{pa 178 -386}\special{pa 177 -384}\special{pa 176 -383}\special{pa 176 -381}%
\special{pa 176 -380}\special{pa 176 -378}\special{pa 176 -376}\special{pa 176 -375}%
\special{pa 176 -373}\special{pa 177 -371}\special{pa 178 -370}\special{pa 179 -369}%
\special{pa 180 -368}\special{pa 182 -367}\special{pa 183 -366}\special{pa 185 -365}%
\special{pa 186 -365}\special{pa 188 -365}\special{pa 190 -365}\special{pa 191 -365}%
\special{pa 193 -365}\special{pa 195 -366}\special{pa 196 -367}\special{pa 198 -368}%
\special{pa 199 -369}\special{pa 200 -370}\special{pa 201 -371}\special{pa 201 -373}%
\special{pa 202 -375}\special{pa 202 -376}\special{pa 202 -378}\special{pa 202 -378}%
\special{sh 1}\special{ip}%
}%
\special{pa   202  -378}\special{pa   202  -380}\special{pa   202  -381}\special{pa   201  -383}%
\special{pa   201  -384}\special{pa   200  -386}\special{pa   199  -387}\special{pa   198  -388}%
\special{pa   196  -389}\special{pa   195  -390}\special{pa   193  -391}\special{pa   191  -391}%
\special{pa   190  -391}\special{pa   188  -391}\special{pa   186  -391}\special{pa   185  -391}%
\special{pa   183  -390}\special{pa   182  -389}\special{pa   180  -388}\special{pa   179  -387}%
\special{pa   178  -386}\special{pa   177  -384}\special{pa   176  -383}\special{pa   176  -381}%
\special{pa   176  -380}\special{pa   176  -378}\special{pa   176  -376}\special{pa   176  -375}%
\special{pa   176  -373}\special{pa   177  -371}\special{pa   178  -370}\special{pa   179  -369}%
\special{pa   180  -368}\special{pa   182  -367}\special{pa   183  -366}\special{pa   185  -365}%
\special{pa   186  -365}\special{pa   188  -365}\special{pa   190  -365}\special{pa   191  -365}%
\special{pa   193  -365}\special{pa   195  -366}\special{pa   196  -367}\special{pa   198  -368}%
\special{pa   199  -369}\special{pa   200  -370}\special{pa   201  -371}\special{pa   201  -373}%
\special{pa   202  -375}\special{pa   202  -376}\special{pa   202  -378}%
\special{fp}%
\special{pa   328  -289}\special{pa   344  -320}%
\special{fp}%
\special{pa   318  -294}\special{pa   333  -326}%
\special{fp}%
\special{pa    50  -467}\special{pa    34  -436}%
\special{fp}%
\special{pa    60  -462}\special{pa    45  -430}%
\special{fp}%
\settowidth{\Width}{$\mathrm{P}\displaystyle(2,1)$}\setlength{\Width}{0\Width}%
\settoheight{\Height}{$\mathrm{P}\displaystyle(2,1)$}\settodepth{\Depth}{$\mathrm{P}\displaystyle(2,1)$}\setlength{\Height}{\Depth}%
\put(2.2500000,1.2500000){\hspace*{\Width}\raisebox{\Height}{$\mathrm{P}\displaystyle(2,1)$}}%
%
\settowidth{\Width}{$\mathrm{Q}\displaystyle(\frac{-2}{5},\frac{11}{5})$}\setlength{\Width}{-1\Width}%
\settoheight{\Height}{$\mathrm{Q}\displaystyle(\frac{-2}{5},\frac{11}{5})$}\settodepth{\Depth}{$\mathrm{Q}\displaystyle(\frac{-2}{5},\frac{11}{5})$}\setlength{\Height}{\Depth}%
\put(-0.6500000,2.4500000){\hspace*{\Width}\raisebox{\Height}{$\mathrm{Q}\displaystyle(\frac{-2}{5},\frac{11}{5})$}}%
%
\settowidth{\Width}{$y=mx$}\setlength{\Width}{-1\Width}%
\settoheight{\Height}{$y=mx$}\settodepth{\Depth}{$y=mx$}\setlength{\Height}{\Depth}%
\put(4.9166667,5.0833333){\hspace*{\Width}\raisebox{\Height}{$y=mx$}}%
%
\special{pa -1181    -0}\special{pa  1181    -0}%
\special{fp}%
\special{pa     0  1181}\special{pa     0 -1181}%
\special{fp}%
\settowidth{\Width}{$x$}\setlength{\Width}{0\Width}%
\settoheight{\Height}{$x$}\settodepth{\Depth}{$x$}\setlength{\Height}{-0.5\Height}\setlength{\Depth}{0.5\Depth}\addtolength{\Height}{\Depth}%
\put(5.0833333,0.0000000){\hspace*{\Width}\raisebox{\Height}{$x$}}%
%
\settowidth{\Width}{$y$}\setlength{\Width}{-0.5\Width}%
\settoheight{\Height}{$y$}\settodepth{\Depth}{$y$}\setlength{\Height}{\Depth}%
\put(0.0000000,5.0833333){\hspace*{\Width}\raisebox{\Height}{$y$}}%
%
\settowidth{\Width}{O}\setlength{\Width}{-1\Width}%
\settoheight{\Height}{O}\settodepth{\Depth}{O}\setlength{\Height}{-\Height}%
\put(-0.0833333,-0.0833333){\hspace*{\Width}\raisebox{\Height}{O}}%
%
\end{picture}}%}
\end{layer}\par
\vspace{15\baselineskip}

下のQRコードからリンクへ飛んでみて。\\
点Pをつまんで動かすと,点Qも動くよ。\\

%% upLaTeXでは,図の挿入にBBdataが必須
\includegraphics[bb=18 18 170 170,width=50mm,clip]{対称移動4QR.png}\\

\url{https://kubo-yuge.github.io/class2019/reflection/reflectionjson.html}\\

\columnbreak

\section{どきどきハート}

スライダの点を動かすと,ハートがドキドキします。\par
\begin{layer}{80}{0}
\putnotese{3}{7}{%%% /Users/kumac21/DataKubo_Mac21/予定とメモなど/予定とメモ(学内)/予定:学内21/学事予定21/オープンキャンパス21秋/出品(久保)/fig/Heart.tex 
%%% Generator=Heart.cdy 
{\unitlength=3mm%
\begin{picture}%
(18,13)(-9,-7)%
\special{pn 8}%
%
\special{pa  -945   591}\special{pa   945   591}%
\special{fp}%
\special{pa -383 591}\special{pa -384 589}\special{pa -384 587}\special{pa -384 586}%
\special{pa -385 584}\special{pa -386 583}\special{pa -387 581}\special{pa -388 580}%
\special{pa -390 579}\special{pa -391 578}\special{pa -393 578}\special{pa -394 577}%
\special{pa -396 577}\special{pa -398 577}\special{pa -399 577}\special{pa -401 578}%
\special{pa -403 578}\special{pa -404 579}\special{pa -405 580}\special{pa -407 581}%
\special{pa -408 583}\special{pa -409 584}\special{pa -409 586}\special{pa -410 587}%
\special{pa -410 589}\special{pa -410 591}\special{pa -410 592}\special{pa -410 594}%
\special{pa -409 596}\special{pa -409 597}\special{pa -408 598}\special{pa -407 600}%
\special{pa -405 601}\special{pa -404 602}\special{pa -403 603}\special{pa -401 603}%
\special{pa -399 604}\special{pa -398 604}\special{pa -396 604}\special{pa -394 604}%
\special{pa -393 603}\special{pa -391 603}\special{pa -390 602}\special{pa -388 601}%
\special{pa -387 600}\special{pa -386 598}\special{pa -385 597}\special{pa -384 596}%
\special{pa -384 594}\special{pa -384 592}\special{pa -383 591}\special{pa -383 591}%
\special{sh 1}\special{ip}%
\special{pa  -383   591}\special{pa  -384   589}\special{pa  -384   587}\special{pa  -384   586}%
\special{pa  -385   584}\special{pa  -386   583}\special{pa  -387   581}\special{pa  -388   580}%
\special{pa  -390   579}\special{pa  -391   578}\special{pa  -393   578}\special{pa  -394   577}%
\special{pa  -396   577}\special{pa  -398   577}\special{pa  -399   577}\special{pa  -401   578}%
\special{pa  -403   578}\special{pa  -404   579}\special{pa  -405   580}\special{pa  -407   581}%
\special{pa  -408   583}\special{pa  -409   584}\special{pa  -409   586}\special{pa  -410   587}%
\special{pa  -410   589}\special{pa  -410   591}\special{pa  -410   592}\special{pa  -410   594}%
\special{pa  -409   596}\special{pa  -409   597}\special{pa  -408   598}\special{pa  -407   600}%
\special{pa  -405   601}\special{pa  -404   602}\special{pa  -403   603}\special{pa  -401   603}%
\special{pa  -399   604}\special{pa  -398   604}\special{pa  -396   604}\special{pa  -394   604}%
\special{pa  -393   603}\special{pa  -391   603}\special{pa  -390   602}\special{pa  -388   601}%
\special{pa  -387   600}\special{pa  -386   598}\special{pa  -385   597}\special{pa  -384   596}%
\special{pa  -384   594}\special{pa  -384   592}\special{pa  -383   591}%
\special{fp}%
\special{pn 24}%
\special{pa   472    -0}\special{pa   484   -15}\special{pa   495   -31}\special{pa   506   -48}%
\special{pa   517   -65}\special{pa   527   -84}\special{pa   538  -103}\special{pa   548  -122}%
\special{pa   558  -143}\special{pa   567  -165}\special{pa   576  -187}\special{pa   584  -210}%
\special{pa   591  -234}\special{pa   598  -259}\special{pa   603  -284}\special{pa   607  -309}%
\special{pa   610  -335}\special{pa   612  -362}\special{pa   611  -388}\special{pa   609  -414}%
\special{pa   605  -440}\special{pa   599  -465}\special{pa   592  -489}\special{pa   582  -513}%
\special{pa   570  -535}\special{pa   556  -556}\special{pa   540  -575}\special{pa   523  -593}%
\special{pa   503  -609}\special{pa   483  -623}\special{pa   461  -635}\special{pa   438  -645}%
\special{pa   414  -653}\special{pa   390  -659}\special{pa   365  -664}\special{pa   340  -667}%
\special{pa   315  -669}\special{pa   289  -669}\special{pa   264  -668}\special{pa   239  -665}%
\special{pa   215  -662}\special{pa   191  -658}\special{pa   168  -652}\special{pa   145  -646}%
\special{pa   122  -640}\special{pa   100  -633}\special{pa    79  -625}\special{pa    58  -617}%
\special{pa    38  -609}\special{pa    19  -600}\special{pa     0  -591}\special{pa   -19  -600}%
\special{pa   -38  -609}\special{pa   -58  -617}\special{pa   -79  -625}\special{pa  -100  -633}%
\special{pa  -122  -640}\special{pa  -145  -646}\special{pa  -168  -652}\special{pa  -191  -658}%
\special{pa  -215  -662}\special{pa  -239  -665}\special{pa  -264  -668}\special{pa  -289  -669}%
\special{pa  -315  -669}\special{pa  -340  -667}\special{pa  -365  -664}\special{pa  -390  -659}%
\special{pa  -414  -653}\special{pa  -438  -645}\special{pa  -461  -635}\special{pa  -483  -623}%
\special{pa  -503  -609}\special{pa  -523  -593}\special{pa  -540  -575}\special{pa  -556  -556}%
\special{pa  -570  -535}\special{pa  -582  -513}\special{pa  -592  -489}\special{pa  -599  -465}%
\special{pa  -605  -440}\special{pa  -609  -414}\special{pa  -611  -388}\special{pa  -612  -362}%
\special{pa  -610  -335}\special{pa  -607  -309}\special{pa  -603  -284}\special{pa  -598  -259}%
\special{pa  -591  -234}\special{pa  -584  -210}\special{pa  -576  -187}\special{pa  -567  -165}%
\special{pa  -558  -143}\special{pa  -548  -122}\special{pa  -538  -103}\special{pa  -527   -84}%
\special{pa  -517   -65}\special{pa  -506   -48}\special{pa  -495   -31}\special{pa  -484   -15}%
\special{pa  -472    -0}\special{pa  -461    14}\special{pa  -450    28}\special{pa  -439    42}%
\special{pa  -428    54}\special{pa  -417    66}\special{pa  -407    78}\special{pa  -396    88}%
\special{pa  -385    99}\special{pa  -375   109}\special{pa  -364   118}\special{pa  -354   128}%
\special{pa  -344   136}\special{pa  -334   145}\special{pa  -325   153}\special{pa  -315   160}%
\special{pa  -305   168}\special{pa  -296   175}\special{pa  -287   182}\special{pa  -277   189}%
\special{pa  -268   195}\special{pa  -259   201}\special{pa  -251   207}\special{pa  -242   213}%
\special{pa  -233   219}\special{pa  -225   225}\special{pa  -216   230}\special{pa  -208   235}%
\special{pa  -199   241}\special{pa  -191   246}\special{pa  -182   251}\special{pa  -174   256}%
\special{pa  -166   261}\special{pa  -157   266}\special{pa  -149   271}\special{pa  -141   276}%
\special{pa  -132   281}\special{pa  -124   286}\special{pa  -115   291}\special{pa  -107   296}%
\special{pa   -98   301}\special{pa   -89   306}\special{pa   -80   311}\special{pa   -71   317}%
\special{pa   -61   322}\special{pa   -52   327}\special{pa   -42   332}\special{pa   -32   338}%
\special{pa   -22   343}\special{pa   -11   349}\special{pa    -0   354}\special{pa    11   349}%
\special{pa    22   343}\special{pa    32   338}\special{pa    42   332}\special{pa    52   327}%
\special{pa    61   322}\special{pa    71   317}\special{pa    80   311}\special{pa    89   306}%
\special{pa    98   301}\special{pa   107   296}\special{pa   115   291}\special{pa   124   286}%
\special{pa   132   281}\special{pa   141   276}\special{pa   149   271}\special{pa   157   266}%
\special{pa   166   261}\special{pa   174   256}\special{pa   182   251}\special{pa   191   246}%
\special{pa   199   241}\special{pa   208   235}\special{pa   216   230}\special{pa   225   225}%
\special{pa   233   219}\special{pa   242   213}\special{pa   251   207}\special{pa   259   201}%
\special{pa   268   195}\special{pa   277   189}\special{pa   287   182}\special{pa   296   175}%
\special{pa   305   168}\special{pa   315   160}\special{pa   325   153}\special{pa   334   145}%
\special{pa   344   136}\special{pa   354   128}\special{pa   364   118}\special{pa   375   109}%
\special{pa   385    99}\special{pa   396    88}\special{pa   407    78}\special{pa   417    66}%
\special{pa   428    54}\special{pa   439    42}\special{pa   450    28}\special{pa   461    14}%
\special{pa   472     0}%
\special{fp}%
\special{pn 8}%
\settowidth{\Width}{$-2\pi$}\setlength{\Width}{-0.5\Width}%
\settoheight{\Height}{$-2\pi$}\settodepth{\Depth}{$-2\pi$}\setlength{\Height}{-\Height}%
\put(-8.0000000,-5.5000000){\hspace*{\Width}\raisebox{\Height}{$-2\pi$}}%
%
\settowidth{\Width}{$-\pi$}\setlength{\Width}{-0.5\Width}%
\settoheight{\Height}{$-\pi$}\settodepth{\Depth}{$-\pi$}\setlength{\Height}{-\Height}%
\put(-4.0000000,-5.5000000){\hspace*{\Width}\raisebox{\Height}{$-\pi$}}%
%
\settowidth{\Width}{$2\pi$}\setlength{\Width}{-0.5\Width}%
\settoheight{\Height}{$2\pi$}\settodepth{\Depth}{$2\pi$}\setlength{\Height}{-\Height}%
\put(8.0000000,-5.5000000){\hspace*{\Width}\raisebox{\Height}{$2\pi$}}%
%
\settowidth{\Width}{$\pi$}\setlength{\Width}{-0.5\Width}%
\settoheight{\Height}{$\pi$}\settodepth{\Depth}{$\pi$}\setlength{\Height}{-\Height}%
\put(4.0000000,-5.5000000){\hspace*{\Width}\raisebox{\Height}{$\pi$}}%
%
\settowidth{\Width}{$0$}\setlength{\Width}{-0.5\Width}%
\settoheight{\Height}{$0$}\settodepth{\Depth}{$0$}\setlength{\Height}{-\Height}%
\put(0.0000000,-5.5000000){\hspace*{\Width}\raisebox{\Height}{$0$}}%
%
\end{picture}}%}
\end{layer}\par
\vspace{11\baselineskip}

下のQRコードからリンクへ飛んでみて。\\

%% upLaTeXでは,図の挿入にBBdataが必須
\includegraphics[bb=18 18 150 150,width=50mm,clip]{HeartQR.png}\\

\url{https://kubo-yuge.github.io/Heartjson.html}\\


\newpage

\end{multicols}

\end{document}
