%%%%%%%%%% 文書毎に変更しない部分を入力しておく %%%%%%%%%%
% upLaTeXコンパイルで対応する。(pLaTeXで太字明朝ができないので。)
% 偶数ページ,奇数ページのヘッダ・フッタを使い分けるために"twoside"が必要。
%\documentclass[dvipdfmx,a4paper,twoside]{jsarticle}
%\documentclass[uplaxex,dvipdfmx,a4paper,twoside]{jsarticle}

\usepackage{ketpic,ketlayer} % ketpicとlayer環境を使うため(下の2つを要求する)
\usepackage{graphicx,xcolor} % png図を挿入するためgraphicxパッケージ & color を 新しいxcolorにする。

\usepackage{amsmath,amssymb,multicol,fancyhdr,mathrsfs}  % AMSパッケージ,多段組パッケージ,ヘッダ制御の追加,筆記体の指定
%{\fboxsep=1pt\fbox{\gt\rule{0em}{0.8em}解説}} % 四角囲みはこれで対応できる。

\usepackage[T1]{fontenc} % iにアクセントを付けるのが上手くいかないので。

\setmargin{24}{24}{11}{39}% ketpicパッケージを使っているのだから,余白はketpicのコマンドで.フッタを考慮。
 \setlength{\columnsep}{6mm}
\renewcommand{\baselinestretch}{.82} % 行送りを0.814--0.829倍に詰めて、48行が収まるように調整。

%%\setmargin{24}{24}{30-vf}{20+vf}% フッタの高さ・位置の分vfだけ調整する。
%%%%%%%%%% ヘッダフッタを制御する。(2ページ以降に重要)
 \pagestyle{fancy} \cfoot{}                                 % ヘッダ・フッタの制御、ページ番号の非印刷
 \setcounter{page}{1}                                       % ページの開始番号を変更
 \renewcommand{\headrulewidth}{0pt}\lhead{}\chead{}\rhead{} % ヘッダの線を引かず、ヘッダを付けない。ページ番号もなし(\thepageを除去)
 \headheight=10mm % ここの値が,上のvfに影響を与える。
 \headsep=8mm % ヘッダと本文が重なるので挿入。

%%%%%%%%%%%%% 弓削紀要様式にするためのプリアンブル部分 %%%%%%%%%%%%%

%%%% 2ページ以降のヘッダ・フッタのためのコマンド %%%%
\newcommand{\secondhead}[2]{
\pagestyle{fancy}\renewcommand{\headrulewidth}{0pt}\renewcommand{\footrulewidth}{0pt}\lfoot{}\cfoot{}\rfoot{}
\chead[#1] {#2} }

%%%%文字コードをUTF-8へ%%%%

%%% 多書体利用の本体(明朝・ゴシックともに2つの太さが使えるようにする) %%%%
\usepackage[deluxe]{otf} % \bfseriesでもゴシックと明朝が使えるようにするため

%%%%%セクションタイトルを中央寄せにして、大きさをノーマルサイズに%%%%
% sectionタイトルの前後は2行あけ、中央寄せにする。section番号とsectionタイトルは続ける。
\makeatletter%% プリアンブルで定義する場合は必須
\renewcommand{\section}{
  \@startsection{section}% #1 見出し
   {1}% #2 見出しのレベル
   {\z@}% #3 横組みの場合,見出し左の空き(インデント量)
   {.5\Cvs \@plus .5\Cdp \@minus.2\Cdp}% #4 見出し上の空き
   {.5\Cvs \@plus.9\Cdp}% #5 見出し下の空き (負の値なら見出し後の空き) 
   {\centering\reset@font\normalsize\gtfamily\bfseries}% #6 見出しの属性(中央揃え&ゴシック・太字)
}% renwecommandの終わり
\makeatother%% プリアンブルで定義する場合は必須
%%%%%%%%%%%%%%%%%%%%%%%%%%%%%%%%%%%%%%%%%%%%

%%%%subsectionタイトルを左寄せ、ノーマルサイズ、前は1行空け、後ろは通常の行送り%%%%
\makeatletter%% プリアンブルで定義する場合は必須
\renewcommand{\subsection}{
  \@startsection{subsection}% #1 見出し
   {2}% #2 見出しのレベル
   {\z@}% #3 横組みの場合,見出し左の空き(インデント量) カウンタ数字を含めた開始位置
   {.5\Cvs \@plus.9\Cdp \@minus.2\Cdp}% #4 見出し上の空き
   {.0\Cvs \@plus.0\Cdp}% #5 見出し下の空き (負の値なら見出し後の空き) 
  {\reset@font\normalsize\gtfamily\bfseries}% #6 見出しの属性(ゴチック&太字)
}% renwecommandの終わり
\makeatother%% プリアンブルで定義する場合は必須
%%%%%%%%%%%%%%%%%%%%%%%%%%%%%%%%%%%%%%%%%%%%55

%%%% section の番号に全角数字を使うことで、紀要の様式に合わせる %%%%
\makeatletter%% プリアンブルで定義する場合は必須
\def\@arabicz#1{%
  \ifcase#1 0\or 1\or 2\or 3\or 4\or 5\or 6\or 7\or 8\or 9\or 10\or
  11\or 12\or 13\or 14\or 15\or 16\or 17\or 18\else\@ctrerr\fi} % (1~10を用意)
  \def\arabicz#1{\expandafter\@arabicz\csname c@#1\endcsname}%123 全角
\makeatother%% プリアンブルで定義する場合は必須

%%%%%%%%%%%% 上で定義した数字を節番号に利用する %%%%%%%%%%%%%%%%%%
\renewcommand{\thesection}{\arabicz{section}{.\hspace{-1.2ex}}} % 節番号をピリオド付きの全角数字にする。
\renewcommand{\thesubsection}{\thesection{\hspace{2ex}\arabicz{subsection}}} % sub節番号をピリオド付きの全角数字にする。

\renewcommand{\thesubsubsection}{\thesubsection{\hspace{.2ex}.\arabicz{subsubsection}}} % subsub節番号をピリオド付きの全角数字にする。

%%%%%%%%%%%% 図のキャプション番号の様式を変更する %%%%%%%%%%%%%%%%%%
\makeatletter
\long\def\@makecaption#1#2{ %
\vskip\abovecaptionskip
\iftdir\sbox\@tempboxa{#1\hskip1zw#2} %
\else\sbox\@tempboxa{\textbf{#1} #2} % ←変更
\fi
\ifdim \wd\@tempboxa >\hsize%
\iftdir #1\hskip1zw#2\relax\par
\else \textbf{#1:} #2\relax\par\fi % ←変更
\else
\global \@minipagefalse
\hbox to\hsize{\hfil\box\@tempboxa\hfil}
\fi
\vskip\belowcaptionskip}
\makeatother

%%%% 行間の指定 %%%%
%この後の行間の指定が、本文の式やsection 位置によって勝手に調整されないように次の2行を加えておく。
\raggedbottom  % 通常の対策
\raggedcolumns % multicol環境に対する対策

%%%%紀要は図の前後を2行ずつ空ける%%%%
% 2段組ページの途中に出力される図と本文の間の余白※未解決
%\abovecaptionskip=0pt \belowcaptionskip=24pt % 図のキャプションの上下を指定、
 \abovecaptionskip=0pt \belowcaptionskip=1.9\Cvs % 図のキャプションの上下を指定、紀要は図の前後を2行ずつ空けるのでキャプションの下だけ2行空ける

%%%%%%%%% 2段組の文章中に画像を挿入するための定義(以下の4行)%%%%%%%%%%
\makeatletter
\newenvironment{tablehere}  {\def\@captype{table}}  {}
\newenvironment{figurehere}  {\def\@captype{figure}}  {}
\makeatother

%%%%%%%%%% 行間の調整めやす %%%%%%%%%%
% subsectionタイトルは前のみ1行あける。1行の中で左寄せ(本文と同じ)。本文とは通常の改行。section番号とsectionタイトルは、1文字あける。
% 図の前後は2行あける。図と図も2行あける。図とsectioタイトル、図とsubsectionタイトルの間は間隔の大きい方とし、2行あける。

% 「Abstract」は、centuryの太字を指定されているので、\bf で代用する。

%%%%%%%%%% 短縮形のためのコマンド %%%%%%%%%%

% \単語Abstract
\newcommand{\単語Abstract}{{
{\bf\fontsize{12pt}{0pt}\selectfont \hfil\ {\hbox to 17mm{A\!\hfil b\!\hfil s\!\hfil t\!\hfil r\!\hfil a\!\hfil c\!\hfil t}}\hfil\ \ }
}}
